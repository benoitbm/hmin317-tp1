% Document description
\documentclass[a4paper,11pt]{report}
\usepackage[utf8]{inputenc}
\usepackage[T1]{fontenc}
\usepackage{lmodern}
\usepackage[francais]{babel}
\usepackage{listings}
\usepackage{graphicx} %Pour inclure les images
\usepackage{float} %Pour plus de précision sur le placement
\usepackage{color}
\usepackage[hidelinks]{hyperref} %Pour les liens dans le PDF
\usepackage{fancyhdr} %En-tête + pieds de page
\usepackage{lastpage}

% Metadata
\title{HMIN317 - Moteur de jeux \\ Compte-rendu TP1}
\author{BOYER Benoît}
\date{Septembre 2017}

\definecolor{mygreen}{rgb}{0,0.6,0}
\definecolor{mygray}{rgb}{0.5,0.5,0.5}
\definecolor{mymauve}{rgb}{0.58,0,0.82}

\lstset{ %
	backgroundcolor=\color{white},   % choose the background color; you must
	%add \usepackage{color} or \usepackage{xcolor}; should come as last argument
	basicstyle=\footnotesize,        % the size of the fonts that are used for
	%the code
	breakatwhitespace=false,         % sets if automatic breaks should only
	%happen at whitespace
	breaklines=true,                 % sets automatic line breaking
	captionpos=b,                    % sets the caption-position to bottom
	commentstyle=\color{mygreen},    % comment style
	deletekeywords={...},            % if you want to delete keywords from the
	%given language
	escapeinside={\%*}{*)},          % if you want to add LaTeX within your code
	extendedchars=true,              % lets you use non-ASCII characters; for
	%8-bits encodings only, does not work with UTF-8
	frame=single,	                   % adds a frame around the code
	keepspaces=true,                 % keeps spaces in text, useful for keeping
	%indentation of code (possibly needs columns=flexible)
	keywordstyle=\color{blue},       % keyword style
	language=Octave,                 % the language of the code
	morekeywords={*,...},           % if you want to add more keywords to the
	%set
	numbers=left,                    % where to put the line-numbers; possible
	%values are (none, left, right)
	numbersep=5pt,                   % how far the line-numbers are from the
	%code
	numberstyle=\tiny\color{mygray}, % the style that is used for the
	%line-numbers
	rulecolor=\color{black},         % if not set, the frame-color may be
	%changed on line-breaks within not-black text (e.g. comments (green here))
	showspaces=false,                % show spaces everywhere adding particular
	%underscores; it overrides 'showstringspaces'
	showstringspaces=false,          % underline spaces within strings only
	showtabs=false,                  % show tabs within strings adding
	%particular underscores
	stepnumber=1,                    % the step between two line-numbers. If
	%it's 1, each line will be numbered
	stringstyle=\color{mymauve},     % string literal style
	tabsize=2,	                   % sets default tabsize to 2 spaces
	title=\lstname                   % show the filename of files included with
	%\lstinputlisting; also try
	%n instead of title
}

%Reglages pour les liens
\hypersetup{
  colorlinks=false,
  linktoc=true, %Pour mettre les liens entre la table des matières et les sections
  pdfauthor = {BOYER Benoît},
  pdftitle = {HMIN317 - Moteur de jeux},
  pdfsubject = {Compte-rendu TP1},
}

%Redifinition du pied de page pour le chapitre uniquement
\fancypagestyle{plain}{%
\renewcommand{\footrulewidth}{1pt}
\fancyfoot[L]{BOYER Benoît}
\fancyfoot[C]{}
\fancyfoot[R]{\textbf{Page \thepage \ sur \pageref{LastPage}}}

\renewcommand{\headrulewidth}{0pt}
\fancyhead[R]{}
\fancyhead[C]{}
\fancyhead[L]{}
}

%Reglages pour en-têtes et pieds de page
\pagestyle{fancy}

\renewcommand{\headrulewidth}{1pt}
\fancyhead[R]{HMIN317 - Moteur de jeux}
\fancyhead[C]{}
\fancyhead[L]{\leftmark}

\renewcommand{\footrulewidth}{1pt}
\fancyfoot[L]{BOYER Benoît}
\fancyfoot[C]{}
\fancyfoot[R]{\textbf{Page \thepage \ sur \pageref{LastPage}}}


% --------------> Document beginning <--------------
\begin{document}

    \section{Question 1}
    \subsection{A quoi servent les classes MainWidget et GeometryEngine ?}
    La classe MainWidget sert à gérer l'affichage et les évènements du programme, en effet si on se fie aux foncions présentes dans le .h :
    \lstinputlisting[language=C++, caption=Fonctions de la classe MainWidget, firstline=75, lastline=85]{../mainwidget.h}
    Les fonctions dans l'ordre permettent :
    \begin{itemize}
    	\item De gérer le clic souris
    	\item De gérer le relachement du clic souris
    	\item De gérer le temps (et faire un mouvement continu, comme par exemple attrapper le cube et lui inculquer un mouvement pour qu'il continue et ralentisse sur le temps)
    	\item D'initialiser la fenêtre OpenGL
    	\item De gérer la modification de la taille de la fenêtre
    	\item De redessiner et mettre à jour le contenu de la fenêtre OpenGL
    	\item D'initialiser les shaders
    	\item D'initialiser les textures (donc les charger et de les préparer pour les utiliser plus tard)
    \end{itemize}   
    \hfill \break
	La classe GeometryEngine est utilisée pour la géométrie du dé présent :
    \lstinputlisting[language=C++, caption=Fonctions de la classe GeometryEngine, firstline=64, lastline=67]{../geometryengine.h}
    Les fonctions présentes servent à dessiner et à initialiser le cube qui, une fois texturé, sera le dé à la fenêtre.

    \pagebreak
    \subsection{A quoi servent les fichiers fshader.glsl et vshader.glsl ?}
    Les fichiers sont dans l'ordre le {\it{fragment shader}} et le {\it{vertex shader}}. Le fragment shader va servir à appliquer la bonne couleur au pixel (en se basant sur la texture fournie), tandis que le vertex shader va calculer la position des vertices par rapport à la fenêtre.
	
	\pagebreak
	\section{Question 2}
	\subsection{Expliquer le fonctionnement des deux fonctions de la classe CubeGeometry.}
	    La fonction {\lstinline{void initCubeGeometry()}} permet d'initialiser le cube en créant dans un premier temps les vertices du cube :
    \lstinputlisting[language=C++, firstline=85, lastline=99, caption=Création de deux faces du cube]{../geometryengine.cpp}
    Pour ensuite dans un second temps, recenser les indices pour faire les triangles des faces des cubes :
    \lstinputlisting[language=C++, firstline=126, lastline=140, caption=Création des indices]{../geometryengine.cpp}
    Et pour terminer, on transfère les données au GPU via des buffers :
    \lstinputlisting[language=C++, firstline=143, lastline=149, caption=Transfert des données via un VBO]{../geometryengine.cpp}
    Une fois terminé, la fonction {\lstinline{void drawCubeGeometry(QOpenGLShaderProgram *program)}} doit être utilisée pour dessiner le cube.\hfill \break
    Dans un premier temps, on va indiquer à OpenGL quels VBO ({\it{vertex shaders}}) utiliser :
    \lstinputlisting[language=C++, firstline=156, lastline=158, caption=Selection des VBO]{../geometryengine.cpp}
	
\end{document}